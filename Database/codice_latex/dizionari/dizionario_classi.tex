\subsection{Dizionario delle Classi}

\vspace{4mm}

\begin{longtable}{|| m{0.2\textwidth} | m{0.35\textwidth} | m{0.45\textwidth} ||}
    \hline\hline
     \textbf{Classe} & \textbf{Descrizione} & \textbf{Attributi} \\
     \hline\hline
     \endfirsthead

     \hline
     \textbf{Classe} & \textbf{Descrizione} & \textbf{Attributi} \\
     \hline
     \endhead

     %Utente
     Utente & L'utente fa l'accesso al sistema o come Compagnia per aggiungere o aggiornare le corsa, o come Cliente per consultare le corse disponibili &
     \textbf{login}(String): login identificativo per accedere all'area riservata della compagnia o del cliente
     
     \textbf{password}(String): password per accedere all'area riservata della compagnia o del cliente\\
     \hline

     %Compagnia
     Compagnia & La compagnia di navigazione offre delle corse in traghetto, aliscafo, motonave o altro tra le varie isole  & 
     \textbf{nome}(String): nome della compagnia

     \textbf{numero}(String): numero di telefono della compagnia

     \textbf{indirizzo}(String): indirizzo di posta elettronica della compagnia
     
     \textbf{sitoWeb}(String): link URL al sito web della compagnia\\
     \hline

     %Porto
     Porto & Luogo in cui le imbarcazioni salpano e attraccano &
     \textbf{comune}(String): nome del comune di appartenenza del porto
     
     \textbf{indirizzo}(String): indirizzo presso il quale è situato il porto
     
     \textbf{numeroTelefono}(String): numero telefonico del servizio informazioni\\
     \hline

     %Scalo
     Scalo & Atrracco al porto di un isola intermedia per consentire ad altri passeggeri di arrivare a destinazione & 
     \textbf{OraAttracco}(Timestamp): ora in cui il natante arriva al porto
     
     \textbf{OraRipartenza}(Timestamp): ora in cui il natante riparte
     
     \textbf{costoPrimaTratta}(float): costo della tratta intermedia che va dal porto di partenza al porto di scalo

     \textbf{costoSecondaTratta}(float): costo della tratta intermedia che va dal porto di scalo al porto di arrivo\\
     \hline

     %CorsaRegolare
     CorsaRegolare & Tratta marittima messa a disposizione da una compagnia che collega due isole (o tre se c'è uno scalo) che si ripeterà per determinati periodi e determinati giorni &   
     \textbf{OrarioPartenza}(Time): ora in cui il natante salpa dal porto di partenza e ha inizio la corsa
     
     \textbf{OrarioArrivo}(Time): ora in cui il natante attracca al porto di arrivo e la corsa ha termine
     
     \textbf{costoIntero}(float): costo base della corsa per un adulto senza bagaglio o veicolo
     
     \textbf{scontoRidotto}(float): percentuale di sconto per un biglietto ridotto

     \textbf{costoBagaglio}(float): sovrapprezzo per i clienti che portano un bagaglio

     \textbf{costoVeicolo}(float): sovrapprezzo per i clienti che imbarcano un veicolo

     \textbf{costoPrevendita}(float): sovrapprezzo per i clienti che acquistano il biglietto in prevendita\\
     \hline

     %CorsaSpecifica
     CorsaSpecifica & Istanza specifica di una corsa regolare &
     \textbf{data}(Date): giorno in cui viene effettuata la corsa
     
     \textbf{postiDispPass}(int): numero di posti passeggeri ancora disponibili
     
     \textbf{postiDispVei}(int): numero di posti veicoli ancora disponibili
     
     \textbf{minutiRitardo}(int): minuti di ritardo della corsa
     
     \textbf{cancellata}(boolean): valore booleano che indica se una corsa è stata cancellata dalla compagnia\\
     \hline

     %Periodo
     Periodo & Periodo dell'anno in cui è attiva una corsa regolare &
     \textbf{dataInizio}(Date): data di inizio del periodo
     
     \textbf{dataFine}(Date): data di fine del periodo
     
     \textbf{giorno}(String): attributo multivalore che indica i giorni in cui la corsa è attiva\\
     \hline
    
     %Natante
     Natante & Categoria di imbarcazione utilizzata dalle compagnie di navigazione &
     \textbf{nome}(String): nome identificativo del natante
     
     \textbf{capienzaPasseggeri}(int): numero massimo di passeggeri che il natante può trasportare\\
     \hline

     %Traghetto
     Traghetto & Specializzazione di Natante dotato di posti per passeggeri e per veicoli &
     \textbf{capienzaVeicoli}(int): numero massimo di veicoli che il natante può trasportare\\
     \hline

     %Aliscafo
     Aliscafo & Specializzazione di Natante dotato di posti solo per passeggeri &
     \\
     \hline

     %Motonave
     Motonave & Specializzazine di Natante dotati di posti solo per passeggeri &
     \\
     \hline

     %Cliente
     Cliente & Utente registrato al sistema che decide di acquistare uno o più biglietti per una o più corse specifiche &
     \textbf{nome}(String): nome del cliente
     
     \textbf{cognome}(String): cognome del cliente\\
     \hline

     %Biglietto
     Biglietto & Biglietto acquistato dal cliente per usufruire di una determinata corsa &
     \textbf{eta}(int): età del passeggero
     
     \textbf{prevendita}(boolean): valore booleano che indica se il cliente ha prenotato o meno il biglietto
     
     \textbf{bagaglio}(boolean): valore booleano che indica se il cliente porta con se o meno un bagaglio
     
     \textbf{prezzo}(int): prezzo totale del biglietto calcolato in base all'età del passeggero e alla presenza del bagaglio e dell'autoveicolo
     
     \textbf{dataAcquisto}(Date): data di acquisto del biglietto\\
     \hline

     %Veicolo
     Veicolo & Informazioni sul veicolo che un cliente vuole imbarcare & 
     \textbf{targa}(String): targa identificativa del veicolo
     
     \textbf{tipo}(String): tipo di veicolo (Automobile, Scooter etc.)\\
     \hline
     
     %AccountSocial
     Account Social & Informazioni sui vari account social di una compagnia &
     \textbf{nomeSocial}(String): nome del social al qual è associato l'account
     
     \textbf{tag}(String): tag dell'account\\
     \hline

     \hline\hline
     
\end{longtable}
