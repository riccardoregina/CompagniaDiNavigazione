\subsection{Creazione del database}
\begin{lstlisting}[style=sqlstyle, title = {Creazione del database}]
    --creazione di un database di nome Navigazione
    create database Navigazione;
\end{lstlisting}

\subsection{Creazione dello schema}
\begin{lstlisting}[style=sqlstyle, title = {Creazione dello schema}]
    --creazione di uno schema di nome Navigazione
    create schema Navigazione;
\end{lstlisting}

\subsection{Creazione dei tipi}
Poiché un natante può essere o un traghetto, o una motonave, o un aliscafo o nessuno di questi tre è stato scelto di definire un nuovo tipo.
\begin{lstlisting}[style=sqlstyle, title = {Creazione del tipo tipoNatante}]
    --creazione del tipo di natante come un'enumerazione dei 4 possibili tipi
    create type tipoNatante as enum ('traghetto', 'motonave', 'aliscafo', 'altro');
\end{lstlisting}

Lo stesso ragionamento è stato fatto per i tipi di veicolo che un cliente può scegliere di imbarcare.
\begin{lstlisting}[style=sqlstyle, title = {Creazione del tipo tipoVeicolo}]
    --creazione del tipo di veicolo come un'enumerazione dei 4 possibili tipi
    create type tipoVeicolo as enum ('automobile', 'motociclo', 'mezzo pesante', 'altro');
\end{lstlisting}

\subsection{Creazione delle tabelle}
La maggior parte dei vincoli di chiave esterna verranno aggiunti in un secondo momento.
\begin{lstlisting}[style=sqlstyle, title = {Creazione della tabella Compagnia}]
    create table Navigazione.Compagnia(
        login text primary key,
        password text not null,
        nome text not null,
        sitoWeb text not null
    );
\end{lstlisting}

\begin{lstlisting}[style=sqlstyle, title = {Creazione della tabella CorsaRegolare}]
    create table Navigazione.CorsaRegolare(
        idCorsa serial primary key,
        PortoPartenza integer not null,
        PortoArrivo integer not null,
        orarioPartenza time not null,
        orarioArrivo time not null,
        costoIntero numeric not null check(costoIntero >= 0),
        scontoRidotto numeric not null check(scontoRidotto >= 0 AND scontoRidotto <=100),
        costoBagaglio numeric default 0 check(costoBagaglio >= 0),
        costoPrevendita numeric default 0 check(costoPrevendita >= 0),
        costoVeicolo numeric default 0 check(costoVeicolo >= 0),
        Compagnia text not null,
        Natante text not null,
        CorsaSup integer not null,
    
        check (PortoArrivo <> PortoPartenza)
    );
\end{lstlisting}

\begin{lstlisting}[style=sqlstyle, title = {Creazione della tabella CorsaSpecifica}]
    create table Navigazione.CorsaSpecifica(
        idCorsa integer,
        data date,
        postiDispPass integer not null check(postiDispPass >= 0),
        postiDispVei integer check(postiDispVei >= 0 or postiDispVei is null),
        minutiRitardo integer not null default 0,
        cancellata boolean not null default 'false',
    
        primary key(idCorsa, data),
        foreign key(idCorsa) references Navigazione.CorsaRegolare(idCorsa)
            on delete cascade       on update cascade
    );
\end{lstlisting}

\begin{lstlisting}[style=sqlstyle, title = {Creazione della tabella Periodo}]
    create table Navigazione.Periodo(
        idPeriodo serial primary key,
        dataInizio date not null,
        dataFine date not null,
        giorni bit(7) not null,
    
        check (dataInizio < dataFine)
    );
\end{lstlisting}

\begin{lstlisting}[style=sqlstyle, title = {Creazione della tabella AttivaIn}]
    create table Navigazione.AttivaIn (
        idCorsa integer,
        idPeriodo integer,
    
        primary key(idCorsa, idPeriodo),
        foreign key(idCorsa) references Navigazione.CorsaRegolare(idCorsa)
            on delete cascade       on update cascade,
    
        foreign key(idPeriodo) references Navigazione.Periodo(idPeriodo)
            on delete cascade       on update cascade
    );
\end{lstlisting}

\begin{lstlisting}[style=sqlstyle, title = {Creazione della tabella Porto}]
    create table Navigazione.Porto(
        idPorto serial primary key,
        comune text not null,
        indirizzo text not null,
        numeroTelefono text not null
    );
\end{lstlisting}

\begin{lstlisting}[style=sqlstyle, title = {Creazione della tabella Scalo}]
    create table Navigazione.Scalo(
        idCorsa integer primary key,
        Porto integer not null,
        orarioAttracco time not null,
        orarioRipartenza time not null,
    
        check(orarioAttracco < orarioRipartenza)
    );
\end{lstlisting}

\begin{lstlisting}[style=sqlstyle, title = {Creazione della tabella Natante}]
    create table Navigazione.Natante(
        nome text primary key,
        Compagnia text not null,
        capienzaPasseggeri integer not null,
        capienzaVeicoli integer,
        tipo tipoNatante not null default 'altro'
    );
\end{lstlisting}

\begin{lstlisting}[style=sqlstyle, title = {Creazione della tabella Cliente}]
    create table Navigazione.Cliente(
        login text primary key,
        password text not null,
        nome text not null,
        cognome text not null
    );
\end{lstlisting}

\begin{lstlisting}[style=sqlstyle, title = {Creazione della tabella Biglietto}]
    create table Navigazione.Biglietto(
        idBiglietto serial primary key,
        idCorsa integer not null,
        data Date not null,
        Cliente text not null,
        Veicolo text,
        prevendita boolean not null default 'false',
        bagaglio boolean not null default 'false',
        prezzo numeric not null check(prezzo >= 0),
        dataAcquisto date not null,
        etaPasseggero integer not null check (etaPasseggero >= 0),
    
        foreign key(idCorsa, data) references Navigazione.CorsaSpecifica(idCorsa, data)
            on delete cascade       on update cascade
    );
\end{lstlisting}

\begin{lstlisting}[style=sqlstyle, title = {Creazione della tabella Veicolo}]
    create table Navigazione.Veicolo(
        targa text primary key,
        tipo tipoVeicolo not null default 'altro',
        Proprietario text not null
    );
\end{lstlisting}

\begin{lstlisting}[style=sqlstyle, title = {Creazione della tabella AccountSocial}]
    create table Navigazione.AccountSocial(
        nomeSocial text,
        tag text,
        Compagnia text not null,
    
        primary key(nomeSocial, tag)
    );
\end{lstlisting}

\begin{lstlisting}[style=sqlstyle, title = {Creazione della tabella Email}]
    create table Navigazione.Email(
        indirizzo text primary key,
        Compagnia text not null
    );
\end{lstlisting}

\begin{lstlisting}[style=sqlstyle, title = {Creazione della tabella Telefono}]
    create table Navigazione.Telefono(
        numero text primary key,
        Compagnia text not null
    );
\end{lstlisting}

\subsection{Creazione dei vincoli di chiave esterna}
\begin{lstlisting}[style=sqlstyle, title = {Aggiunti vincoli di chiave esterna per la tabella CorsaRegolare}]
    alter table Navigazione.CorsaRegolare
        add constraint corsaFKcompagnia 
            foreign key (Compagnia) references Navigazione.Compagnia(login)
                on delete cascade       on update cascade;
    
    alter table Navigazione.CorsaRegolare
        add constraint corsaFKnatante
            foreign key (Natante) references Navigazione.Natante(nome)
                on delete cascade   on update cascade;
    
    alter table Navigazione.CorsaRegolare
        add constraint corsaFKportoPartenza
            foreign key (PortoPartenza) references Navigazione.Porto(idPorto)
                on delete cascade       on update cascade;
    
    alter table Navigazione.CorsaRegolare
        add constraint corsaFKportoArrivo
            foreign key (PortoArrivo) references Navigazione.Porto(idPorto)
                on delete cascade       on update cascade;
    
    alter table Navigazione.CorsaRegolare
        add constraint corsaFKcorsaSup
            foreign key (CorsaSup) references Navigazione.CorsaRegolare(idCorsa)
                on delete cascade       on update cascade;
\end{lstlisting}

\begin{lstlisting}[style=sqlstyle, title = {Aggiunti vincoli di chiave esterna per la tabella Scalo}]
    alter table Navigazione.Scalo
        add constraint scaloFKcorsa
            foreign key (idCorsa) references Navigazione.CorsaRegolare(idCorsa)
                on delete cascade       on update cascade;
    
    alter table Navigazione.Scalo
        add constraint scaloFKporto
            foreign key (Porto) references Navigazione.Porto(idPorto)
                on delete cascade       on update cascade;
\end{lstlisting}

\begin{lstlisting}[style=sqlstyle, title = {Aggiunto vincolo di chiave esterna per la tabella Natante}]
    alter table Navigazione.Natante
        add constraint natanteFKcompagnia
            foreign key (Compagnia) references Navigazione.Compagnia(login)
                on delete cascade       on update cascade;
\end{lstlisting}

\begin{lstlisting}[style=sqlstyle, title = {Aggiunti vincoli di chiave esterna per la tabella Biglietto}]
    alter table Navigazione.Biglietto
        add constraint bigliettoFKcliente
            foreign key (Cliente) references Navigazione.Cliente(login)
                on delete cascade       on update cascade;
    
    alter table Navigazione.Biglietto
        add constraint bigliettoFKveicolo
            foreign key (Veicolo) references Navigazione.Veicolo(targa)
                on delete set null      on update cascade;
\end{lstlisting}

\begin{lstlisting}[style=sqlstyle, title = {Aggiunto vincolo di chiave esterna per la tabella Veicolo}]
    alter table Navigazione.Veicolo
        add constraint veicoloFKproprietario
            foreign key (Proprietario) references Navigazione.Cliente(login)
                on delete cascade       on update cascade;
\end{lstlisting}

\begin{lstlisting}[style=sqlstyle, title = {Aggiunto vincolo di chiave esterna per la tabella AccountSocial}]
    alter table Navigazione.AccountSocial
        add constraint accountFKcompagnia
            foreign key (Compagnia) references Navigazione.Compagnia(login)
                on delete cascade       on update cascade;
\end{lstlisting}

\begin{lstlisting}[style=sqlstyle, title = {Aggiunto vincolo di chiave esterna per la tabella Email}]
    alter table Navigazione.Email
        add constraint emailFKcompagnia
            foreign key (Compagnia) references Navigazione.Compagnia(login)
                on delete cascade       on update cascade;
\end{lstlisting}

\begin{lstlisting}[style=sqlstyle, title = {Aggiunto vincolo di chiave esterna per la tabella Telefono}]
    alter table Navigazione.Telefono
        add constraint telefonoFKcompagnia
            foreign key (Compagnia) references Navigazione.Compagnia(login)
                on delete cascade       on update cascade;
\end{lstlisting}

\subsection{Trigger e trigger function}
\label{sec:ImplementazioneTrigger}
La descrizione dei trigger è riportata nella sezione \hyperref[sec:VincoliInterRelazionali]{Vincoli Inter-Relazionali}.

\begin{lstlisting}[style=sqlstyle]
-- trigger per aggiornare i posti (per passeggeri) disponibili per una corsa specifica
create function aggiornapostipasseggero() returns trigger
    language plpgsql
as
$$
declare
    v_corsa_sup navigazione.corsaregolare.corsasup%type;
begin
    select corsasup into v_corsa_sup
    from navigazione.corsaregolare
    where idcorsa = new.idcorsa;

    raise notice 'corsa sup %', v_corsa_sup;
    --se il biglietto acquistato e' per una corsa principale 
    if v_corsa_sup is null then 
        --aggiorna i posti disponibili anche per le sottocorse
        update navigazione.corsaspecifica
        set postidisppass = postidisppass - 1
        where data = new.data and idcorsa in (select CR.idcorsa
                                              from navigazione.corsaregolare as CR
                                              where CR.corsasup = new.idcorsa or CR.idcorsa = new.idcorsa);
    else --se invece il biglietto acquistato e' per una sottotratta 
        --aggiorna i posti per la sottotratta in questione
        update navigazione.corsaspecifica
        set postidisppass = postidisppass - 1
        where idcorsa = new.idCorsa and data = new.data;
        
        --aggiorna i posti per la corsa principale
        update navigazione.corsaspecifica
        set postidisppass = postidisppass - 1
        where idcorsa = v_corsa_sup and data = new.data;
    end if;
    
    return null;
end;
$$;
----------
create trigger triggeraggiornapostipasseggero
    after insert
    on biglietto
    for each row
execute procedure aggiornapostipasseggero();
\end{lstlisting}

\begin{lstlisting}[style = sqlstyle]
-- trigger per aggiornare i posti (per veicoli) disponibili per una corsa specifica
create function aggiornapostiveicolo() returns trigger
    language plpgsql
as
$$
declare
    v_corsa_sup navigazione.corsaregolare.corsasup%type;
begin
    select corsasup into v_corsa_sup
    from navigazione.corsaregolare
    where idcorsa = new.idcorsa;

    --se il biglietto acquistato e' per una corsa principale 
    if v_corsa_sup is null then
        --aggiorna i posti disponibili anche per le sottocorse
        update navigazione.corsaspecifica
        set postidispvei = postidispvei - 1
        where data = new.data and idcorsa in (select CR.idcorsa
                                              from navigazione.corsaregolare as CR
                                              where CR.corsasup = new.idcorsa or CR.idcorsa = new.idcorsa);
    else --se invece il biglietto acquistato e' per una sottotratta 
        --aggiorna i posti per la sottotratta in questione
        update navigazione.corsaspecifica
        set postidispvei = postidispvei - 1
        where idcorsa = new.idCorsa and data = new.data;

        --aggiorna i posti per la corsa principale
        update navigazione.corsaspecifica
        set postidispvei = postidispvei - 1
        where idcorsa = v_corsa_sup and data = new.data;
    end if;

    return null;
end;
$$;
----------
create trigger triggeraggiornapostiveicolo
    after insert
    on biglietto
    for each row
    when (new.veicolo IS NOT NULL)
execute procedure aggiornapostiveicolo();
\end{lstlisting}

\begin{lstlisting}[style = sqlstyle]
-- all'inserimento di uno scalo per una corsaregolare, questo trigger si occupa di generare le sottocorse e di attivarle nei periodi
--in cui e' attiva la corsa principale.
create function aggiungicorsescalo() returns trigger
    language plpgsql
as
$$
declare
    v_record navigazione.corsaregolare%rowtype;
    v_ultimo_idcorsa navigazione.corsaregolare.idcorsa%type;
    --seleziona tutti i periodi in cui e' attiva la corsa principale
    cur_periodi cursor for
        select idperiodo
        from navigazione.attivain
        where idcorsa = new.idcorsa;
begin
    --seleziona in corsaRegolare l'intera tupla corrispondente alla corsa principale
    select * into v_record
    from navigazione.corsaregolare
    where idcorsa = new.idcorsa;

    --inserisce in corsaRegolare la prima sottotratta che parte dal porto di partenza della corsa principale
    --e arriva al porto di scalo
    insert into navigazione.corsaregolare(portopartenza, portoarrivo, orariopartenza, orarioarrivo,
                                          costointero, scontoridotto, costobagaglio, costoprevendita,
                                          costoveicolo, compagnia, natante, corsasup)
    values
        (v_record.portopartenza, new.porto, v_record.orariopartenza, new.orarioAttracco, v_record.costointero,
         v_record.scontoridotto, v_record.costobagaglio, v_record.costoprevendita, v_record.costoveicolo,
         v_record.compagnia, v_record.natante, new.idcorsa);

    --seleziona quest'ultima corsa inserita
    select idcorsa into v_ultimo_idcorsa
    from navigazione.corsaregolare
    order by idcorsa desc
    limit 1;

    --e l'attiva in tutti i periodi in cui era attiva la corsa principale
    for p in cur_periodi loop
        insert into navigazione.attivain
        values (v_ultimo_idcorsa, p.idperiodo);
    end loop;

    --inserisce in corsaRegolare la seconda sottotratta che parte dal porto di scalo 
    --e arriva al porto di arrivo della corsa principale
    insert into navigazione.corsaregolare(portopartenza, portoarrivo, orariopartenza, orarioarrivo,
                                          costointero, scontoridotto, costobagaglio, costoprevendita,
                                          costoveicolo, compagnia, natante, corsasup)
    values
        (new.porto, v_record.portoarrivo, new.orarioRipartenza, v_record.orarioarrivo, v_record.costointero,
         v_record.scontoridotto, v_record.costobagaglio, v_record.costoprevendita, v_record.costoveicolo,
         v_record.compagnia, v_record.natante, new.idcorsa);

    --seleziona quest'ultima corsa inserita
    select idcorsa into v_ultimo_idcorsa
    from navigazione.corsaregolare
    order by idcorsa desc
    limit 1;

    --e l'attiva in tutti i periodi in cui era attiva la corsa principale
    for p in cur_periodi loop
            insert into navigazione.attivain
            values (v_ultimo_idcorsa, p.idperiodo);
    end loop;
    
    return new;
end;
$$;
----------
create trigger generatrattescalotrigger
    after insert
    on scalo
    for each row
execute procedure aggiungicorsescalo();
\end{lstlisting}

\begin{lstlisting}[style = sqlstyle]
-- trigger per attivare nei periodi della corsa principale anche le sue sottocorse
create function attivasottocorse() returns trigger
    language plpgsql
as
$$
declare
    --seleziona le due sottocorse della corsa in questione
    cur_sottocorse cursor for
        select idcorsa
        from navigazione.corsaregolare
        where corsasup = new.idcorsa;
begin
    --e attiva ognuna nel nuovo periodo inserito
    for t in cur_sottocorse loop
        insert into navigazione.attivain
        values (t.idcorsa, new.idperiodo);
    end loop;

    return new;
end;
$$;
----------
create trigger attivasottocorsetrigger
    after insert
    on attivain
    for each row
execute procedure attivasottocorse();
\end{lstlisting}

\begin{lstlisting}[style=sqlstyle]
--trigger che quando viene cambiato l'orario di arrivo della corsa principale
--aggiorna l'orario di arrivo anche della sottocorsa che parte dal porto di scalo
create function cambiaorarioarrivoinsottocorsa() returns trigger
    language plpgsql
as
$$
declare
    sottocorsa navigazione.corsaregolare%rowtype;
begin
    --seleziona la sottocorsa che arriva allo stesso orario della corsa principale
    select * into sottocorsa
    from navigazione.corsaregolare
    where corsasup = old.idcorsa and orarioarrivo = old.orarioarrivo;

    --aggiorna l'orario di arrivo
    update navigazione.corsaregolare
    set orarioarrivo = new.orarioarrivo
    where idcorsa = sottocorsa.idcorsa;

    return new;
end;
$$;
----------
create trigger cambiaorarioarrivoinsottocorsa
    after update
        of orarioarrivo
    on corsaregolare
    for each row
    --quando e' una corsa principale
    when (new.corsasup IS NULL AND new.orarioarrivo IS NOT NULL)
execute procedure cambiaorarioarrivoinsottocorsa();

\end{lstlisting}

\begin{lstlisting}[style=sqlstyle]
--trigger che quando viene cambiato l'orario di partenza della corsa principale
--aggiorna l'orario di partenza anche della sottocorsa che parte dallo stesso porto
create function cambiaorariopartenzainsottocorsa() returns trigger
    language plpgsql
as
$$
declare
    sottocorsa navigazione.corsaregolare%rowtype;

begin
    --seleziona la sottocorsa che parte allo stesso orario della corsa principale
    select * into sottocorsa
    from navigazione.corsaregolare
    where corsasup = old.idcorsa and orariopartenza = old.orariopartenza;

    --aggiorna l'orario di partenza
    update navigazione.corsaregolare
    set orariopartenza = new.orariopartenza
    where idcorsa = sottocorsa.idcorsa;

    return new;
end;
$$;
----------
create trigger cambiaorariopartenzainsottocorsa
    after update
        of orariopartenza
    on corsaregolare
    for each row
    when (new.corsasup IS NULL AND new.orariopartenza IS NOT NULL)
execute procedure cambiaorariopartenzainsottocorsa();

\end{lstlisting}

\begin{lstlisting}[style=sqlstyle]
--trigger per eliminare le corse specifiche di una corsa regolare non piu' attiva in un periodo
--ed eliminare i periodi che non sono attaccati a nessuna corsa
create function cancellacorseinperiodo() returns trigger
    language plpgsql
as
$$
declare
    v_data_inizio navigazione.periodo.datainizio%type;
    v_data_fine navigazione.periodo.dataFine%type;
    v_data_corrente date;
    v_giorni bit(7);
    v_day integer;
    v_offset integer;
    --trovo i periodi che non sono attaccati a nessuna corsa
    cur_periodi cursor for
        select P.idperiodo
        from navigazione.periodo as P
        where P.idperiodo not in (select idperiodo 
                                  from navigazione.attivain);
begin
    select datainizio, datafine, giorni into v_data_inizio, v_data_fine, v_giorni
    from navigazione.periodo
    where idperiodo = old.idperiodo;

    --ricava un numero da 0 a 6 corrispondente al giorno della settimana in cui inizia il periodo
    -- 0 se domenica, 1 se lunedi' ... fino a 6 se sabato
    v_day := extract(dow from v_data_inizio::timestamp);
    --per ogni bit della stringa di bit che indica i giorni di attivita'
    for i in 0..6 loop
        --ogni iterazione del for-loop parte con v_data_corrente uguale alla data di inizio
        v_data_corrente := v_data_inizio;
        --se l'i-esimo bit e' 1
        if get_bit(v_giorni, i) = 1 then
            v_offset := (i - v_day);
            if v_offset < 0 then
                while v_offset < 0 loop
                        v_offset := v_offset + 7;
                end loop;
            else
                v_offset := v_offset % 7;
            end if;
            --aggiunge l'offset alla data corrente affinche' la data corrente
            --si trovi nella prima data maggiore o uguale la data di inizio,
            --che abbia come giorno della settimana l'i-esimo giorno
            v_data_corrente := v_data_corrente + v_offset;
            --per tutte le date con quel giorno incluse nel periodo
            while v_data_corrente <= v_data_fine loop
                --elimina la corsa specifica per quella data
                delete from navigazione.corsaspecifica
                where idcorsa = old.idcorsa AND data = v_data_corrente;
                --incrementa la data alla settimana successiva
                v_data_corrente := v_data_corrente + 7;
            end loop;
        end if;
    end loop;

    --elimina i periodi che non sono attaccati a nessuna corsa
    for p in cur_periodi loop
        delete from navigazione.periodo
        where idperiodo = p.idperiodo;
    end loop;
    
    return null;
end;
$$;
----------
create trigger cancellacorse
    after delete
    on attivain
    for each row
execute procedure cancellacorseinperiodo();

\end{lstlisting}

\begin{lstlisting}[style = sqlstyle]
--Trigger per eliminare le sottocorse dopo che e' stato eliminato uno scalo
create function eliminacorsescalo() returns trigger
    language plpgsql
as
$$
begin
    delete from navigazione.corsaregolare
    where corsasup = old.idcorsa AND (portopartenza = old.porto OR portoarrivo = old.porto);
    
    return null;
end;
$$;
----------
create trigger eliminatrattescalotrigger
    after delete
    on scalo
    for each row
execute procedure eliminacorsescalo();

\end{lstlisting}

\begin{lstlisting}[style = sqlstyle]
--Trigger per eliminare l'altra sottocorsa dopo che ne e' stata eliminata una, poiche' non puo' esistere una senza l'altra. 
create function eliminasottocorse() returns trigger
    language plpgsql
as
$$
declare
    v_altra_corsa navigazione.corsaregolare.idcorsa%type;
begin
    if old.corsasup is not null then
        --seleziona la sottocorsa rimanente
        select idcorsa into v_altra_corsa
        from navigazione.corsaregolare
        where corsasup = old.corsasup;

        if v_altra_corsa is not null then
            --e la cancella
            delete from navigazione.corsaregolare
            where idcorsa = v_altra_corsa;
        end if;
        
        --elimina la tupla in scalo che aveva generato le due sottocorse
        delete from navigazione.scalo
        where idcorsa = old.corsasup;
    end if;
    return null;
end;
$$;
----------
create trigger eliminaaltrasottocorsa
    after delete
    on corsaregolare
    for each row
    --quando la corsa eliminata e' una sottocorsa
    when (old.corsasup IS NOT NULL)
execute procedure eliminasottocorse();

\end{lstlisting}

\begin{lstlisting}[style=sqlstyle]
-- trigger per la generazione delle corse specifiche in tutte le date presenti nel periodo di attivazione
create function generacorsespecifiche() returns trigger
    language plpgsql
as
$$
declare
    v_data_inizio date;
    v_data_corrente date;
    v_data_fine date;
    v_giorni bit(7);
    v_natante navigazione.natante.nome%type;
    v_cap_pass integer;
    v_cap_veic integer;
    v_day integer;
    v_offset integer;
begin
    --seleziona le date di inizio e fine e i giorni del periodo
    select dataInizio, dataFine, giorni into v_data_inizio, v_data_fine, v_giorni
    from navigazione.periodo
    where idperiodo = new.idperiodo;

    --seleziona il natante che verra' utilizzato
    select natante into v_natante
    from navigazione.corsaRegolare
    where idcorsa = new.idcorsa;

    --seleziona la capienza passeggeri e veicoli del natante
    select capienzaPasseggeri, capienzaVeicoli into v_cap_pass, v_cap_veic
    from navigazione.natante
    where nome = v_natante;

    --ricava un numero da 0 a 6 corrispondente al giorno della settimana in cui inizia il periodo
    -- 0 se domenica, 1 se lunedi' ... fino a 6 se sabato
    v_day := extract(dow from v_data_inizio::timestamp);
    --per ogni bit della stringa di bit che indica i giorni di attivita'
    for i in 0..6 loop
        --ogni iterazione del for-loop parte con v_data_corrente uguale alla data di inizio
        v_data_corrente := v_data_inizio;
        --se l'i-esimo bit e' 1
        if get_bit(v_giorni, i) = 1 then
            v_offset := (i - v_day);
            if v_offset < 0 then
                while v_offset < 0 loop
                    v_offset := v_offset + 7;
                end loop;
            else
                v_offset := v_offset % 7;
            end if;
            --aggiunge l'offset alla data corrente affinche' la data corrente
            --si trovi nella prima data maggiore o uguale la data di inizio,
            --che abbia come giorno della settimana l'i-esimo giorno
            v_data_corrente := v_data_corrente + v_offset;
            --per tutte le date con quel giorno incluse nel periodo
            while v_data_corrente <= v_data_fine loop
                --aggiunge la corsa specifica per quella data
                insert into navigazione.corsaSpecifica
                values(new.idcorsa, v_data_corrente, v_cap_pass, v_cap_veic, 0, false);
                --incrementa la data alla settimana successiva
                v_data_corrente := v_data_corrente + 7;
            end loop;
        end if;
    end loop;
    
    return new;
end;
$$;
----------
create trigger generacorse
    after insert
    on attivain
    for each row
execute procedure generacorsespecifiche();

\end{lstlisting}

\begin{lstlisting}[style=sqlstyle]
-- trigger per cancellare le sottocorse figlie (quando viene cancellata una corsa principale), e la
-- sottocorsa sorella (quando viene cancellata una sottocorsa).
create function propagacancellazione() returns trigger
    language plpgsql
as
$$
declare
    thisData date := old.data;
    CorsaRegolareRiferita navigazione.corsaregolare%rowtype;
    isCancellata boolean;
    it1 cursor for select *
                   from  navigazione.corsaregolare
                   where corsasup = CorsaRegolareRiferita.idcorsa;
    it2 cursor for select *
                    from navigazione.corsaregolare
                    where corsasup = CorsaRegolareRiferita.corsasup and idcorsa <> CorsaRegolareRiferita.idcorsa;
begin
    --seleziona la corsa regolare della corsa in questione
    select * into CorsaRegolareRiferita from navigazione.corsaregolare where idcorsa = old.idcorsa;
    --se e' una corsa principale
    if (CorsaRegolareRiferita.corsasup is null) then --potrebbe avere delle sottocorse
        --per entrambe le sottocorse
        for x in it1 loop
            select cancellata into isCancellata
            from navigazione.corsaspecifica
            where idcorsa = x.idcorsa and data = thisData;
            --se non sono gia' cancellate le cancella
            if isCancellata = false then
                update navigazione.corsaspecifica
                set cancellata = true
                where idcorsa = x.idcorsa and data = thisData;
            end if;
        end loop;
    else --se invece e' una sottocorsa ha sicuramente una sorella
        for x in it2 loop
            --e cancella entrambe
            select cancellata into isCancellata
            from navigazione.corsaspecifica
            where idcorsa = x.idcorsa and data = thisData;
            if isCancellata = false then
                update navigazione.corsaspecifica
                set cancellata = true
                where idcorsa = x.idcorsa and data = thisData;
            end if;
        end loop;
    end if;
    return new;
end;
$$;
----------
create trigger propagacancellazione
    after update
        of cancellata
    on corsaspecifica
    for each row
    when (new.cancellata = true)
execute procedure propagacancellazione();
\end{lstlisting}

\subsection{Procedure}
\begin{lstlisting}[style=sqlstyle]
--Procedura per calcolare gli incassi di una corsa regolare in un determinato arco di tempo
create procedure calcolaincassicorsainperiodo(IN thisidcorsa integer, IN ini_periodo date, IN fin_periodo date, OUT incasso double precision)
    language plpgsql
as
$$
begin
    select sum(prezzo) into incasso
    from navigazione.biglietto
    where idcorsa = thisidcorsa and data between ini_periodo and fin_periodo;
end;
$$;

\end{lstlisting}

\begin{lstlisting}[style=sqlstyle]
--Procedura per fare in modo che quando venga creato uno scalo in una corsa regolare, per ogni data
--in cui e' attiva, le sottocorse create abbiano gli stessi posti disponibili della corsa madre.
create procedure aggiornapostisottocorse(IN idcorsa_in integer)
    language plpgsql
as
$$
declare
    --seleziona le sottocorse
    cur_sottocorse cursor for
        select CR.idcorsa
        from navigazione.corsaregolare as CR
        where CR.corsasup = idCorsa_in;

    --seleziona i dati che vanno aggiornati nelle sottocorse
    cur_postidisp cursor for
        select CS.idcorsa, CS.data, CS.postidisppass, CS.postidispvei, CS.cancellata
        from navigazione.corsaspecifica as CS
        where CS.idcorsa = idCorsa_in;
begin
    for k in cur_postidisp loop
        for i in cur_sottocorse loop
            --aggiorna i valori delle sottocorse
            update navigazione.corsaspecifica
            set postidisppass = k.postidisppass, postidispvei = k.postidispvei, cancellata = k.cancellata
            where idcorsa = i.idcorsa AND data = k.data;
        end loop;
    end loop;
end;
$$;
\end{lstlisting}
\newpage

\subsection{Dizionario dei vincoli}
\subsubsection{Vincoli Intra-Relazionali}

\begin{longtable}{|| m{0.3\textwidth} | m{0.7\textwidth} ||}
    \hline\hline
     \textbf{Vincolo} & \textbf{Descrizione} \\ [1ex]
     \hline\hline
     \endfirsthead

      \textit{accountsocial\_pkey} & Vincolo di chiave primaria  \\ [1ex]
      \hline

      \textit{accountfkcompagnia} & Vincolo di chiave esterna. Se viene eliminata la compagnia vengono eliminati anche tutti i suoi account social. \\ [1ex]
     \hline\hline

     \caption*{Tabella AccountSocial} \\
\end{longtable}

\begin{longtable}{|| m{0.4\textwidth} | m{0.6\textwidth} ||}
    \hline\hline
     \textbf{Vincolo} & \textbf{Descrizione} \\ [1ex]
     \hline\hline
     \endfirsthead

      \textit{attivain\_pkey} & Vincolo di chiave primaria  \\ [1ex]
      \hline

      \textit{attivain\_idcorsa\_fkey} & Vincolo di chiave esterna. Se viene eliminata la corsa regolare vengono eliminate anche le tuple associate in AttivaIn. \\ [1ex]
     \hline

     \textit{attivain\_idperiodo\_fkey} & Vincolo di chiave esterna. Se viene eliminato un periodo vengono eliminate anche le tuple associate in AttivaIn. \\ [1ex]
     \hline\hline

     \caption*{Tabella AttivaIn} \\
\end{longtable}

\begin{longtable}{|| m{0.4\textwidth} | m{0.6\textwidth} ||}
    \hline\hline
     \textbf{Vincolo} & \textbf{Descrizione} \\ [1ex]
     \hline\hline
     \endfirsthead

      \textit{biglietto\_pkey} & Vincolo di chiave primaria  \\ [1ex]
      \hline

      \textit{biglietto\_idcorsa\_data\_fkey} & Vincolo di chiave esterna. Se viene eliminata una corsa specifica anche i biglietti associati verranno eliminati \\ [1ex]
     \hline

     \textit{bigliettofkcliente} & Vincolo di chiave esterna. Se viene eliminato un cliente anche i biglietti associati verrano eliminati \\ [1ex]
     \hline

     \textit{bigliettofkveicolo} & Vincolo di chiave esterna. Se viene eliminato un veicolo il valore in biglietto sarà settato a null \\ [1ex]
     \hline

     \textit{biglietto\_etapasseggero\_check} & Controlla che l'età del passeggero sia un numero non negativo. \\ [1ex]
     \hline

     \textit{biglietto\_prezzo\_check} & Controlla che il prezzo del biglietto sia un numero non negativo. \\ [1ex]
     \hline\hline

     \caption*{Tabella Biglietto} \\
\end{longtable}

\begin{longtable}{|| m{0.3\textwidth} | m{0.7\textwidth} ||}
    \hline\hline
     \textbf{Vincolo} & \textbf{Descrizione} \\ [1ex]
     \hline\hline
     \endfirsthead

      \textit{cliente\_pkey} & Vincolo di chiave primaria. \\ [1ex]
     \hline\hline

     \caption*{Tabella Cliente} \\
\end{longtable}

\begin{longtable}{|| m{0.3\textwidth} | m{0.7\textwidth} ||}
    \hline\hline
     \textbf{Vincolo} & \textbf{Descrizione} \\ [1ex]
     \hline\hline
     \endfirsthead

      \textit{compagnia\_pkey} & Vincolo di chiave primaria. \\ [1ex]
     \hline\hline

     \caption*{Tabella Compagnia} \\
\end{longtable}

\begin{longtable}{|| m{0.5\textwidth} | m{0.5\textwidth} ||}
    \hline\hline
     \textbf{Vincolo} & \textbf{Descrizione} \\ [1ex]
     \hline\hline
     \endfirsthead

      \textit{corsaregolare\_pkey} & Vincolo di chiave primaria. \\ [1ex]
        \hline

      \textit{corsafkcompagnia} & Vincolo di chiave esterna. Se viene eliminata la compagnia verrano eliminati anche tutte le sue corse regolari. \\ [1ex]
     \hline

     \textit{corsafkcorsasup} & Vincolo di chiave esterna. Se viene eliminata la corsa principale anche le sottocorse verranno eliminate. \\ [1ex]
     \hline

     \textit{corsafkportoarrivo} & Vincolo di chiave esterna. Se viene eliminato il porto di arrivo, verranno eliminate tutte le corse che arrivano in quel porto. \\ [1ex]
     \hline

     \textit{corsafkportopartenza} & Vincolo di chiave esterna. Se viene eliminato il porto di partenza, verranno eliminate tutte le corse che partivano da quel porto. \\ [1ex]
     \hline

    \textit{corsafknatante} & Vincolo di chiave esterna. Se viene eliminato il natante, verrano eliminate tutte le corse che usavano quel natante. \\ [1ex]
     \hline

     \textit{corsaregolare\_check} & Controlla che il porto di arrivo sia diverso dal porto di partenza. \\ [1ex]
     \hline

     \textit{corsaregolare\_costobagaglio\_check} & Controlla che il costo del bagaglio sia un numero non negativo. \\ [1ex]
     \hline

     \textit{corsaregolare\_costointero\_check} & Controlla che il costo intero della corsa sia un numero non negativo. \\ [1ex]
     \hline

     \textit{corsaregolare\_costoprevendita\_check} & Controlla che il costo della prevendita sia un numero non negativo. \\ [1ex]
     \hline

     \textit{corsaregolare\_costoveicolo\_check} & Controlla che il costo del veicolo sia un numero non negativo. \\ [1ex]
     \hline

     \textit{corsaregolare\_scontoridotto\_check} & Controlla che la percentuale di sconto sia un numero compreso tra $0$ e $100$. \\ [1ex]      
     \hline\hline

     \caption*{Tabella CorsaRegolare} \\
\end{longtable}

\begin{longtable}{|| m{0.5\textwidth} | m{0.5\textwidth} ||}
    \hline\hline
     \textbf{Vincolo} & \textbf{Descrizione} \\ [1ex]
     \hline\hline
     \endfirsthead

      \textit{corsaspecifica\_pkey} & Vincolo di chiave primaria. \\ [1ex]
      \hline

      \textit{corsaspecifica\_idcorsa\_fkey} & Vincolo di chiave esterna. Se viene eliminata una corsa regolare, tutte le corse specifiche associato verranno eliminate. \\ [1ex]
      \hline
        
      \textit{corsaspecifica\_postidisppass\_check} & Controlla che i posti passeggeri disponibili sia un numero non negativo \\ [1ex]
     \hline

     \textit{corsaspecifica\_postidisppass\_check} & Controlla che i posti veicoli disponibili siano un numero non negativo \\ [1ex]      
     \hline\hline

     \caption*{Tabella CorsaSpecifica} \\
\end{longtable}
\newpage

\begin{longtable}{|| m{0.3\textwidth} | m{0.7\textwidth} ||}
    \hline\hline
     \textbf{Vincolo} & \textbf{Descrizione} \\ [1ex]
     \hline\hline
     \endfirsthead

      \textit{email\_pkey} & Vincolo di chiave primaria. \\ [1ex]
      \hline

      \textit{emailfkcompagnia} & Vincolo di chiave esterna. Se una compagnia viene eliminata anche tutte le sue email verrano eliminate. \\ [1ex]      
     \hline\hline

     \caption*{Tabella Email} \\
\end{longtable}

\begin{longtable}{|| m{0.3\textwidth} | m{0.7\textwidth} ||}
    \hline\hline
     \textbf{Vincolo} & \textbf{Descrizione} \\ [1ex]
     \hline\hline
     \endfirsthead

      \textit{natante\_pkey} & Vincolo di chiave primaria. \\ [1ex]
      \hline

      \textit{natantefkcompagnia} & Vincolo di chiave esterna. Se una compagnia viene eliminata anche i natanti da essa posseduta verranno eliminati. \\ [1ex]      
     \hline\hline

     \caption*{Tabella Natante} \\
\end{longtable}

\begin{longtable}{|| m{0.3\textwidth} | m{0.7\textwidth} ||}
    \hline\hline
     \textbf{Vincolo} & \textbf{Descrizione} \\ [1ex]
     \hline\hline
     \endfirsthead

      \textit{periodo\_pkey} & Vincolo di chiave primaria. \\ [1ex]
      \hline

      \textit{periodo\_check} & Controlla che la data di inizio periodo sia minore o uguale alla data di fine periodo. \\ [1ex]      
     \hline\hline

     \caption*{Tabella Periodo} \\
\end{longtable}

\begin{longtable}{|| m{0.3\textwidth} | m{0.7\textwidth} ||}
    \hline\hline
     \textbf{Vincolo} & \textbf{Descrizione} \\ [1ex]
     \hline\hline
     \endfirsthead

      \textit{porto\_pkey} & Vincolo di chiave primaria. \\ [1ex]    
     \hline\hline

     \caption*{Tabella Porto} \\
\end{longtable}

\begin{longtable}{|| m{0.3\textwidth} | m{0.7\textwidth} ||}
    \hline\hline
     \textbf{Vincolo} & \textbf{Descrizione} \\ [1ex]
     \hline\hline
     \endfirsthead

      \textit{scalo\_pkey} & Vincolo di chiave primaria. \\ [1ex]
      \hline

      \textit{scalofkcorsa} & Vincolo di chiave esterna. Se una corsa viene eliminata, anche lo scalo associato viene eliminato. \\ [1ex]
      \hline

      \textit{scalofkporto} & Vincolo di chiave esterna. Se un porto viene eliminato, anche gli scali in quel porto verranno eliminati. \\ [1ex]
      \hline

      \textit{scalo\_check} & Controlla che l'orario di attracco sia minore dell'orario di ripartenza. \\ [1ex]     
     \hline\hline

     \caption*{Tabella Scalo} \\
\end{longtable}

\begin{longtable}{|| m{0.3\textwidth} | m{0.7\textwidth} ||}
    \hline\hline
     \textbf{Vincolo} & \textbf{Descrizione} \\ [1ex]
     \hline\hline
     \endfirsthead

      \textit{telefono\_pkey} & Vincolo di chiave primaria. \\ [1ex]
      \hline

      \textit{telefonofkcompagnia} & Vincolo di chiave esterna. Se una compagnia viene eliminata anche i suoi numeri telefonici verranno eliminati.\\ [1ex]      
     \hline\hline

     \caption*{Tabella Telefono} \\
\end{longtable}

\begin{longtable}{|| m{0.3\textwidth} | m{0.7\textwidth} ||}
    \hline\hline
     \textbf{Vincolo} & \textbf{Descrizione} \\ [1ex]
     \hline\hline
     \endfirsthead

      \textit{veicolo\_pkey} & Vincolo di chiave primaria. \\ [1ex]
      \hline

      \textit{veicolofkproprietario} & Vincolo di chiave esterna. Se un cliente viene eliminato anche tutti i suoi veicoli verranno eliminati.\\ [1ex]      
     \hline\hline

     \caption*{Tabella Veicolo} \\
\end{longtable}

\subsubsection{Vincoli Inter-Relazionali}
\label{sec:VincoliInterRelazionali}

L'implementazione dei trigger è riportata nella sezione \hyperref[sec:ImplementazioneTrigger]{Trigger e trigger function}.

\begin{longtable}{|| m{0.45\textwidth} | m{0.55\textwidth} ||}
    \hline\hline
     \textbf{Trigger} & \textbf{Descrizione} \\ [1ex]
     \hline\hline
     \endfirsthead

      \textit{attivaSottoCorseTrigger} & Dopo aver inserito una tupla in \textit{AttivaIn}, anche le eventuali sottocorse della corsa inserita verranno inserite in \textit{AttivaIn}. \\ [1ex]
      \hline

      \textit{cancellaCorseTrigger} & Dopo aver eliminato una una tupla in \textit{AttivaIn}, vengono eliminate tutte le corse specifiche relative a quella corsa, in tutte le date appartenenti a quel periodo \\ [1ex]
      \hline

      \textit{generaCorse} & Dopo aver inserito una tupla in \textit{AttivaIn}, per tutte le date appartenti al periodo in questione verranno inserite delle corse specifiche. \\ [1ex]
      \hline

      \textit{triggerAggiornaPostiPasseggero} & Dopo aver inserito una tupla in \textit{Biglietto}, il trigger aggiorna i posti disponibili per quella corsa specifica, inoltre se la corsa in questione è una corsa principale, il trigger aggiornerà i posti anche per le sottocorse, se invece la corsa è una sottocorsa, il trigger aggionerà i posti anche per la corsa principale ma non per l'altra sottocorsa. \\ [1ex]
      \hline

      \textit{triggerAggiornaPostiVeicolo} & Dopo aver inserito una tupla in \textit{Biglietto}, il trigger aggiorna i posti disponibili per quella corsa specifica, inoltre se la corsa in questione è una corsa principale, il trigger aggiornerà i posti anche per le sottocorse, se invece la corsa è una sottocorsa, il trigger aggionerà i posti anche per la corsa principale ma non per l'altra sottocorsa. Il trigger si attiva solo quando il valore di \textit{veicolo} è diverso da null. \\ [1ex]
      \hline

      \textit{cambiaOrarioArrivoInSottocorsa} & Dopo aver modificato il valore di \textit{orarioArrivo} in \textit{CorsaRegolare}, il trigger modificherà il valore anche nelle eventuali sottocorse. Il trigger si attiva quindi solo quando \textit{corsaSup} è null. \\ [1ex]
      \hline

      \textit{cambiaOrarioPartenzaInSottocorsa} & Dopo aver modificato il valore di \textit{orarioPartenza} in \textit{CorsaRegolare}, il trigger modificherà il valore anche nelle eventuali sottocorse. Il trigger si attiva quindi solo quando \textit{corsaSup} è null. \\ [1ex]
      \hline

      \textit{eliminaAltraSottocorsa} & Dopo aver eliminato una tupla in \textit{CorsaRegolare} e quando \textit{corsaSup} è diverso da null, quindi si è appena eliminata una sottocorsa, il trigger andrà ad eliminare anche l'altra sottocorsa che ha lo stesso valore di \textit{corsaSup}. \\ [1ex]
      \hline

      \textit{propagaCancellazione} & Dopo aver settato a \textit{true} il valore di \textit{cancellata} in \textit{CorsaSpecifica}, il trigger andrà a settare a \textit{true} anche il valore di \textit{cancellata} delle corse specifiche che sono sottocorse della corsa in questione. Se invece la corsa cancellata è una sottocorsa, il trigger andrà a cancellare anche l'altra sottocorsa, dal momento che una sottocorsa non può esistere senza l'altra. \\ [1ex]
      \hline

      \textit{eliminaTratteScalo} & Dopo aver eliminato una tupla in \textit{Scalo}, il trigger elimina le sottocorse della corsa in questione. \\ [1ex]
      \hline

      \textit{generaTratteScalo} & Dopo aver inserito una tupla in \textit{Scalo}, il trigger genera le due sottocorse, una avrà come porto di partenza lo stesso della corsa inserita e come porto di arrivo il porto di scalo, l'altra invece avrà come porto di partenza il porto di scalo e come porto di arrivo il porto di arrivo della corsa inserita. \\ [1ex]          
     \hline\hline
\end{longtable}














